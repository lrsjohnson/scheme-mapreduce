\documentclass{article}

\usepackage{graphicx}
\usepackage{amssymb}
\usepackage{amsmath}
\usepackage{amsfonts}
\usepackage{verbatim}
\usepackage{indentfirst}
\usepackage{amsthm}
\usepackage{framed}
\usepackage{enumitem}

\usepackage{titling}

%%%%%
% Setup
%%%%%
\newcommand{\p}{\mathbb{P}}
\newcommand{\e}{\mathbb{E}}

% Margins
\setlength{\textwidth}{6.5in}
\setlength{\oddsidemargin}{0in}
\setlength{\evensidemargin}{0in}

%\setlength{\droptitle}{-0.5in}

\def\squarebox#1{\hbox to #1{\hfill\vbox to #1{\vfill}}}
\newcommand{\qedbox}
{\vbox{\hrule\hbox{\vrule\squarebox{.667em}\vrule}\hrule}}
%\newcommand{\qed}               {\nopagebreak\mbox{}\hfill\qedbox\smallskip}

\title{MapReduce System (MRS)\\
{\large 6.945 Final Project - Spring 2014}}
\author{Lars Johnson (larsj@mit.edu) \\ Gurtej Kanwar (gurtej@mit.edu)}
\date{May 12, 2014}

%%%%%
% Document
%%%%%
\begin{document}
\pagestyle{myheadings}
\maketitle

\markright{MapReduce System - 6.945 Final Project - Lars Johnson}

\section{Overview}

\section{Objecives}

Our project 

\begin{itemize}

\item Explore applying concepts from 6.945 to systems programming

\item Emphasize flexibility by creating a clean abstraction

\item Enable operations on data sets in a simple and fundamental way
\end{itemize}

\section{Context}

\subsection{MapReduce Background}

\subsection{Extending MapReduce}

\section{Our System}

\subsection{Key Ideas}

Build a graph of data sets connected by operations

Feed data into data sets and it will be processed in a distributed manner across a worker pool

Abstraction system to allow for streaming implementations

Provide programmers with a combinator-like family of reusable parts


\subsection{System Architecture}

Our MapReduce System is comprised of four main components (see diagram
1). On the top level, a set of user operations (mrs:create-data-set)
(mrs:map...), (mrs:reduce...), (mrs:feed-data ...) etc. enable a user
of the system to specify a dataflow network and feed data into
it. Executing such specifications from within a
(mrs:run-computation...) call directs a master process to allocate the
threads and spawn a distriobutor task for each requested
computation. These distributors coordinate execution of the function
and will spawn a number of worker threads to perform the exeuction.

\subsection{Communication Details}

A key component of developing such a system is managing communications
betweent various tasks. We used conspire:threads to provide
multi-tasking behavior for the tasks and used both non-blocking pipes
and data-sets to coordinate between systems. 

The non-blocking pipes mirror the standard pipes used conspire:threads
and consist of alocked queue. They are used as our demo means
for communiczating between small groups over long ranges.

Our main ``Data Set'' objects were implemented as multi-reader
multi-writer queues of elements.

Each of these data set elements were created as a 

\subsection{Flexible Implementations}

Many of these components were designed with a focus on flexibility to
admit several possible different implementations.

1) For the inter=process communication, our pipe structure

2) For intra-process communication, we developed a handful of different

\subsection{Combinator System}

Finally, we are excited to be able to provide a combinator-like system
for defining and building distributed systems computation graphcs. The
syntax for defining a network closely resembles that from the
propagator system in which a user allocates a number of cells
representing data-sets, followed by specifying operations that
transform the contents of cell data sets into another another.

\section{Conceptual Challenges}

In addition to the technical difficulties of exploring the
multitasking and distributed systems paradigm in more depth, our
project involved a handful of conceptual challenges, particularly
in dealing with the propagation of ``done'' signals to properly handle
reductions and aggregations on data streams.

\subsection{Done Propagation}

...

\subsection{Detecting Completion}

... Return to the user? Depends-on

\section{Future Work}

...??

\end{document}
